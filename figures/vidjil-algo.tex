\begin{figure}[H]
    \centering
    \begin{tikzpicture}[node distance=0cm, thick, font=\small]

    % Première ligne
    \node[
        rectangle, 
        fill=blue!30, 
        text width=4cm, 
        align=left,
        minimum width=4cm, 
        minimum height=0.6cm
        ] at (0,0) (v_predicted1) {VVVVVV};

    \node[right=of v_predicted1] (sub) {\textcolor{red}{X}};

    \node[
        rectangle, 
        right=of sub, 
        fill=blue!30, 
        text width=3cm, 
        align=left,
        minimum width=3cm, 
        minimum height=0.6cm
        ] (v_predicted2) {VVVVVVVVVV};

    \node[
        rectangle, 
        right=2cm of v_predicted2, 
        fill=orange!30, 
        text width=5cm, 
        align=left,
        minimum width=5cm, 
        minimum height=0.6cm
        ] (j_predicted) {JJJJJJJJJJJJJJJJJJ};

    % Fenêtre w
    \draw[draw=violet, thick]
        ($(v_predicted2.west)!0.5!(v_predicted2.east) + (0,0.35)$) rectangle
        ($(j_predicted.west)!0.5!(j_predicted.east) + (0,-0.35)$);

    % Annotations
    \node[anchor=north west] at ([yshift=-1pt]v_predicted1.south west) 
        {\scriptsize \textbf{\textcolor{blue}{V prédit}}};

    \draw[thick, blue] ($(v_predicted1.west)+(0,-0.8)$) -- ($(v_predicted2.east)+(0,-0.8)$) 
        node[anchor=north west] at ([yshift=-0.5cm]v_predicted1.south west) 
        {\scriptsize \textcolor{blue}{V réel}};

    \node[anchor=north east] at ([yshift=-1pt]j_predicted.south east) 
        {\scriptsize \textbf{\textcolor{orange}{J prédit}}};

    \draw[thick, orange] ($(j_predicted.west)+(0,-0.8)$) -- ($(j_predicted.east)+(0,-0.8)$) 
        node[anchor=north east] at ([yshift=-0.5cm]j_predicted.south east) 
        {\scriptsize \textcolor{orange}{J réel}};
        
    \node[above=3pt of sub] {\scriptsize \textbf{\textcolor{red}{Substitution}}};

    \node at ($(v_predicted2)!0.5!(j_predicted)+(0,0.6)$) 
        {\scriptsize \textbf{\textcolor{violet}{Fenêtre extraite w}}};

    % Deuxième ligne
    \begin{scope}[yshift=-2cm]

    \node[
        rectangle, 
        fill=blue!30, 
        text width=4cm, 
        align=left,
        minimum width=4cm, 
        minimum height=0.6cm
        ] at (0,0) (v_predicted1b) {VVVVVVVVVVVV};

    \node[
        rectangle, 
        right=-0.5 cm of v_predicted1b, 
        fill=blue!30,
        minimum width=2.5cm, 
        minimum height=0.6cm
        ] (v_predicted2b) {};

    \node[right=of v_predicted2b] (subb) {\textcolor{red}{X}};

    \node[
        rectangle, 
        right=3.7cm of v_predicted2b, 
        fill=orange!30, 
        text width=5cm, 
        align=left,
        minimum width=5cm, 
        minimum height=0.6cm
        ] (j_predictedb) {JJJJJJJJJJJJJJJJJJ};

    % Fenêtre w
    \draw[draw=violet, thick]
        ($(v_predicted2b.west)!0.5!(v_predicted2b.east) + (0,0.35)$) rectangle
        ($(j_predictedb.west)!0.3!(j_predictedb.east) + (0,-0.35)$);

    % Annotations
    \node[anchor=north west] at ([yshift=-1pt]v_predicted1b.south west) 
        {\scriptsize \textbf{\textcolor{blue}{V prédit}}};

    \draw[thick, blue] ($(v_predicted1b.west)+(0,-0.8)$) -- ($(v_predicted2b.east)+(1.7,-0.8)$) 
        node[anchor=north west] at ([yshift=-0.5cm]v_predicted1b.south west) 
        {\scriptsize \textcolor{blue}{V réel}};

    \node[anchor=north east] at ([yshift=-1pt]j_predictedb.south east) 
        {\scriptsize \textbf{\textcolor{orange}{J prédit}}};

    \draw[thick, orange] ($(j_predictedb.west)+(0,-0.8)$) -- ($(j_predictedb.east)+(0,-0.8)$) 
        node[anchor=north east] at ([yshift=-0.5cm]j_predictedb.south east) 
        {\scriptsize \textcolor{orange}{J réel}};
        
    \node[above=3pt of subb] {\scriptsize \textbf{\textcolor{red}{Substitution}}};

    \node at ($(v_predicted2b)!0.6!(j_predictedb)+(0,0.6)$) {\scriptsize \textbf{\textcolor{violet}{Fenêtre extraite w}}};

    \end{scope}

    \end{tikzpicture}
    \caption{Illustration du fonctionnement de l'heuristique de 
    \textit{vidjil-algo} sur deux lectures successives. Adapté de \citeauthor{giraudFastMulticlonalClusterization2014b}}
    \label{fig:vidjil-algo}
\end{figure}
