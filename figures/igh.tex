\begin{figure}[H]
    \centering
    \begin{tikzpicture}[node distance=0cm, thick, font=\tiny]

    % Segments avec séquences nucléotidiques
    \node[rectangle, draw, fill=blue!20, minimum width=6cm, minimum height=0.8cm, anchor=west] (v_full) at (0,0)
        {\shortstack{\textbf{VH3-30} \\ \dots ATGAATAGCCTGAGGCCT\textcolor{red}{G}AGGACACGGCTG\textcolor{red}{TG}TATTAC\textcolor{red}{T}GTAC\sout{A}}};

    \node[rectangle, draw, right=of v_full, fill=red!20, dashed, minimum width=0.6cm, minimum height=0.8cm] (n1_full)
        {\shortstack{\textbf{N} \\ AAAGGGGTGGAC}};

    \node[rectangle, draw, right=of n1_full, fill=green!20, minimum width=1.5cm, minimum height=0.8cm] (d_full)
        {\shortstack{\textbf{D3-3} \\ \sout{CG}ATGTTTGGAGT\sout{GGTT}}};

    \node[rectangle, draw, right=of d_full, fill=red!20, dashed, minimum width=0.6cm, minimum height=0.8cm] (n2_full)
        {\shortstack{\textbf{N} \\ TCG}};

    \node[rectangle, draw, right=of n2_full, fill=orange!20, minimum width=2cm, minimum height=0.8cm] (j_full)
        {\shortstack{\textbf{J4*02} \\ \sout{ACTGGGGCCAGG}GAACCCT\dots}};

    \node[rectangle, draw, right=of j_full, fill=gray!20, minimum width=1cm, minimum height=0.8cm] (c_full) {C};

    % Régions 3' et 5'
    \node[left=0.2cm of v_full.west] {\scriptsize \textbf{5'}};
    \node[right=0.2cm of c_full.east] {\scriptsize \textbf{3'}};

    % Régions fonctionnelles
    \draw[thick, blue] ($(v_full.west)+(-1,-0.6)$) -- ++(1,0) node[midway, below=2pt] {\scriptsize Leader};
    \draw[thick, blue] ($(v_full.west)+(0.7,-0.6)$) -- ++(1,0) node[midway, below=2pt] {\scriptsize FR1};
    \draw[thick, blue] ($(v_full.west)+(2,-0.6)$) -- ++(1,0) node[midway, below=2pt] {\scriptsize FR2};
    \draw[thick, blue] ($(v_full.west)+(4.7,-0.6)$) -- ++(1,0) node[midway, below=2pt] {\scriptsize FR3};

    \draw[thick, red] ($(v_full.east)+(-0.5,+0.6)$) -- ($(j_full.west)+(0.5,+0.6)$)
        node[midway, above=1.5pt] {\scriptsize \textbf{CDR3}};

    % Nom du réarrangement
    \node[below=1.5cm of n1_full.center, anchor=center, font=\large] {
        \textcolor{blue!60!black}{\texttt{IGHV3-30 1}}\,/%
        \textcolor{red!70!black}{\texttt{AAAGGGGTGGAC(12)}}\,/%
        \textcolor{green!70!black}{\texttt{2}}\,%
        \textcolor{green!70!black}{\texttt{D3-3 4}}\,/%
        \textcolor{red!70!black}{\texttt{TCG}}\,/%
        \textcolor{orange!80!black}{\texttt{12}}
        \textcolor{orange!80!black}{\texttt{J4*02}}
    };
    
    \end{tikzpicture}
    \caption{Illustration d'un réarrangement de la chaîne lourde des immunoglobulines \gls{igh}.
    Les bases rayées correspondent aux nucélotides excisés lors de la recombinaison. 
    Les bases en rouge sont celles mutées par rapport aux séquences germinales lors du 
    processus d'hypermutation somatique.}
    \label{fig:igh}
\end{figure}
