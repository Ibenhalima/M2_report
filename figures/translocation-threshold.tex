
\begin{figure}[H]
    \centering
    \begin{tikzpicture}
        \begin{axis}[
            width=15cm,
            height=8cm,
            xlabel={Seuil $t$},
            xmode=log,
            extra x ticks={6},
            extra x tick labels={$6$},
            ylabel={Nombre de patients},
            legend pos=north east,
            grid=major,
        ]
        \addplot[
            color=Purple,
            mark=*,
        ] table [x=t, y=T, col sep=comma] {data/threshold_analysis.csv};
        \addlegendentry{$T(t)$}

        \addplot[
            color=Goldenrod,
            mark=square*,
        ] table [x=t, y=T+, col sep=comma] {data/threshold_analysis.csv};

        \draw[dashed, thick, color=Brown] (axis cs:6,-100) -- (axis cs:6,1000);
        \node at (axis cs:6.5,450) [anchor=west, font=\small, text=Brown] {seuil $t^*$ optimal};

        \addlegendentry{$T_+(t)$}
        \end{axis}
    \end{tikzpicture}
    \caption{
        Courbes de \textcolor{Purple}{$T(t)$} et \textcolor{Goldenrod}{$T_+(t)$} à partir des 1178 échantillons du \gls{chu}.
        Le \textcolor{Brown}{seuil optimal $t^*$ est indiqué par la ligne pointillée}, il est choisi de façon à maximiser \textcolor{Purple}{$T(t)$} 
        tout en contraignant \textcolor{Goldenrod}{$T_+(t)$} en dessous de 10.
        }
    \label{fig:translocation-threshold}
\end{figure}
