\chapter{Préambule}

Ce stage a été réalisé conjointement entre l'équipe Bonsai du \gls{cristal}, 
dirigée par le Pr. Mikaël Salson, également encadrant de ce travail, 
et l'équipe de bioinformatique du \gls{chu}, coordonnée par Martin Figeac. 
Il s'inscrit dans une collaboration de longue date entre ces deux équipes, 
et notamment en interaction étroite avec le laboratoire d'Hématologie du Pr. Claude Preudhomme, 
qui bénéficie de l'expertise du laboratoire Bonsai dans le domaine de la bioinformatique 
des séquences.

\vspace{1em}

L'équipe Bonsai, au sein du laboratoire \gls{cristal}, regroupe des chercheurs et enseignants-chercheurs
spécialisés en bioinformatique, algorithmique et modélisation des données biologiques. 
Elle développe notamment des méthodes innovantes pour l'analyse des séquences génomiques, 
via des approches d'algorithmique du texte et structures d'index, avec un accent particulier 
sur les méthodes sans alignements. 

\vspace{1em}

L'équipe de bioinformatique du \gls{chu} de Lille, quant à elle, 
s'inscrit dans une dynamique translationnelle forte entre recherche et application clinique. 
Elle travaille en étroite collaboration avec les thématiques médicales, notamment en onco-hématologie, 
pour concevoir et appliquer des outils bioinformatiques performants dans le suivi et le diagnostic des patients.
Elle regroupe 12 ingénieurs et techniciens concourant au développement et maintien de solutions bioinformatiques,
pour l'analyse de données de \gls{ngs}, mais aussi la gestion d'un parc informatique riche doté 
d'un \gls{hpc} et la conception d'outils et solutions utilisateurs et développeurs.

\vspace{1em}

Cette collaboration a donné naissance à plusieurs projets et outils 
largement diffusés et utilisés dans la communauté scientifique, 
et plus particulièrement dans le domaine médical, où ils font aujourd'hui figure de référence. 
Parmi ceux-ci, on peut citer l'outil \textit{FiLT3r}, permettant la détection 
et la quantification des variants de type \gls{itd} du gène \textit{FLT3} \cite{boudryFrugalAlignmentfreeIdentification2022}, 
ainsi que la plateforme \textit{Vidjil} \cite{duezVidjilWebPlatform2016}, dédiée à l'étude des répertoires immunitaires B et T.

\vspace{1em}

C'est concernant ce dernier outil que s'inscrit ce stage, dans la continuité de travaux précédents,
avec l'objectif d'adapter et de développer de nouvelles fonctionnalités via \textit{Vidjil} 
pour l'analyse de la \gls{mrd} dans les hémopathies malignes lymphoïdes, en particulier dans le myélome multiple.