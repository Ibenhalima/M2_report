\chapter{Problématique et objectifs}

Ainsi, l'objectif de ce travail a été de développer un protocole d'analyse via \textit{Vidjil} 
pour l'étude de la \gls{mrd} dans le \gls{mm}. Dans un premier temps, l'objectif a été 
de mettre en place un protocole reproductible permettant d'identifier les réarrangements clonaux 
à partir d'ADN génomique, extrait de prélèvements médullaires. Cette phase a servi de référence 
pour caractériser le réarrangement \gls{vdj} dominant propre à chaque patient, et constituer une 
signature clonale utilisable pour le suivi ultérieur, sur ADN génomique et \gls{adnc}.

\vspace{1em}

Dans un second temps, l'objectif (bien que non atteint à ce jour, mais qui aura guidé les décisions entreprises) 
était de rechercher ces réarrangements dans l'\gls{adnc} issu de prélèvements sanguins. L'analyse de ce type d'échantillons 
soulève plusieurs défis techniques et bioinformatiques : d'une part, la très faible concentration d'\gls{adnc} dans le 
plasma nécessite des méthodes de quantification sensibles et fiables ; d'autre part la fragmentation naturelle du \gls{adnc}, 
rend plus difficile l'identification des réarrangements clonaux, souvent de l'ordre de 500 pb contre seulement 150 pb pour l'\gls{adnc}. 

\vspace{1em}

Un autre pan de ce travail a également consisté à adapter l'analyse faite par \textit{Vidjil} à ces spécificités, 
en évaluant la pertinence des paramètres de détection utilisés par \textit{Vidjil} pour l'\gls{adnc} (ou l'ADN génomique 
traité comme tel) et en développant des outils de traitement personnalisés pour l'exploration des résultats. 

\vspace{1em}

Enfin, une étude des performances analytiques du protocole a été menée, avec une attention particulière portée à la 
sensibilité (détection de faibles taux de \gls{mrd}) et à la spécificité (éviction des signaux non clonaux ou artefactuels), 
afin d'évaluer la faisabilité de cette approche dans un environnement clinique pour le suivi thérapeutique.

