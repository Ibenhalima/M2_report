\chapter{Introduction}

\textit{Vidjil} est une plateforme open-source \cite{giraudFastMulticlonalClusterization2014}
permettant l'analyse de données de séquençage à haut débit pour l'étude du répertoire des gènes \gls{vdj} des 
cellules lymphoïdes. Ce processus de recombinaison somatique se produit au sein des précurseurs
B et T lors de la lymphopoïèse, et permet de générer un vaste répertoire d'anticorps et de récepteurs,
capables à leur tour de reconnaître une grande variété d'antigènes \cite{jonesTamingTransposonVDJ2004}.

\section{Répertoire des gènes \gls{vdj}}

Ce processus de recombinaison est complexe, et implique plusieurs étapes successives de réarrangement des segments
\gls{vdj} au sein des loci des \glspl{ig} et des \glspl{tcr} \cite{rothVDJRecombinationMechanism2014}.
Il débute par la sélection aléatoire de segments D et J, suivie de la sélection d'un segment V, au sein des chaînes
dites « lourdes », sous l'action des recombinases \gls{vdj}. Un mécanisme identique est observé pour les chaînes « légères »,
avec une recombinaison directe des segments V et J, sans étape D. À la suite de chaque recombinaison, des régions de diversité N 
de quelques nucléotides sont insérées aléatoirement entre les segments, contribuant à la diversité potentielle du répertoire 
(\autoref{fig:vdj}).

\begin{figure}[H]
    \centering
    \begin{tikzpicture}[
        node distance=0.3cm and 0cm,
        >=Stealth,
        thick,
        font=\small,
        scale=0.95,
        every node/.style={scale=0.95}
        ]

    % Chaîne lourde : segments germinaux 
    \foreach \name/\label/\i in {v1/V1/0, v2/V2/1, v3/V3/2, v4/V\dots/3, vn/Vn/4} {
        \node[rectangle, draw, fill=blue!20, minimum width=1.3cm, minimum height=0.8cm]
            (\name) at ($(0,0)+(1.4*\i,0)$) {\label};
    }

    \foreach \name/\label/\i in {d1/D1/0, d2/D\dots/1, dn/DN/2} {
        \node[rectangle, draw, fill=green!30, minimum width=0.4cm, minimum height=0.8cm]
            (\name) at ($(vn.east)+(1.1+0.8*\i,0)$) {\label};
    }

    \foreach \name/\label/\i in {j1/J1/0, j2/J2/1, j3/J\dots/2, jn/JN/3} {
        \node[rectangle, draw, fill=orange!30, minimum width=0.6cm, minimum height=0.8cm]
            (\name) at ($(dn.east)+(1.1+0.8*\i,0)$) {\label};
    }

    \node[rectangle, draw, right=0.5cm of jn, fill=gray!30, minimum width=2cm, minimum height=0.8cm] (c) {C};

    \draw[thick] ($(vn.east)+(0,-0.1)$) -- ($(d1.west)+(0,-0.1)$);
    \draw[thick] ($(dn.east)+(0,-0.1)$) -- ($(j1.west)+(0,-0.1)$);
    \draw[thick] ($(jn.east)+(0,-0.1)$) -- ($(c.west)+(0,-0.1)$);

    \node[anchor=west] at ($(v1.north west)+(0,0.4cm)$) {\textbf{Segments germinaux - chaîne lourde}};

    \node at ($(v1.west)+(-0.3,0)$) {\textbf{5'}};
    \node at ($(c.east)+(0.3,0)$) {\textbf{3'}};

    % Recombinaison D-J 
    \node[rectangle, draw, below=2cm of c, fill=gray!30, minimum width=2cm, minimum height=0.8cm] (c_bis) {C};
    \node[rectangle, draw, left=0.5cm of c_bis, fill=orange!30, minimum width=0.6cm, minimum height=0.8cm] (j_recomb) {J};
    \node[rectangle, draw, left=of j_recomb, fill=red!20, dashed, minimum width=0.3cm, minimum height=0.8cm] (n1) {N};
    \node[rectangle, draw, left=of n1, fill=green!30, minimum width=0.4cm, minimum height=0.8cm] (d_recomb) {D};

    \foreach \name/\label/\i in {v1_bis/V1/0, v2_bis/V2/1, v3_bis/V3/2, v4_bis/V\dots/3, vn_bis/Vn/4} {
        \node[rectangle, draw, fill=blue!20, minimum width=1.3cm, minimum height=0.8cm]
            (\name) at ($(d_recomb.east)+(-7.5+1.4*\i,0)$) {\label};
    }

    \draw[thick] ($(vn_bis.east)+(0,-0.1)$) -- ($(d_recomb.west)+(0,-0.1)$);
    \draw[thick] ($(j_recomb.east)+(0,-0.1)$) -- ($(c_bis.west)+(0,-0.1)$);

    \draw[->, thick] (d2.south) -- (d_recomb.north);
    \draw[->, thick] (j2.south) -- (j_recomb.north);

    \node[anchor=west] at ($(v1_bis.north west)+(0,0.4cm)$) {\textbf{Recombinaison D-J}};

    % Recombinaison V-DJ 
    \node[rectangle, draw, below=2cm of c_bis, fill=gray!30, minimum width=2cm, minimum height=0.8cm] (c_bis2) {C};
    \node[rectangle, draw, left=of c_bis2, fill=orange!30, minimum width=0.6cm, minimum height=0.8cm] (j_final) {J};
    \node[rectangle, draw, left=of j_final, fill=red!20, dashed, minimum width=0.3cm, minimum height=0.8cm] (n2) {N};
    \node[rectangle, draw, left=of n2, fill=green!30, minimum width=0.4cm, minimum height=0.8cm] (d_final) {D};
    \node[rectangle, draw, left=of d_final, fill=red!20, dashed, minimum width=0.3cm, minimum height=0.8cm] (n3) {N};
    \node[rectangle, draw, left=of n3, fill=blue!20, minimum width=1.3cm, minimum height=0.8cm] (v_final) {V};

    \draw[thick] ($(j_final.east)+(0,-0.1)$) -- ($(c_bis2.west)+(0,-0.1)$);
    \draw[->, thick] (v2_bis.south) -- (v_final.north);

    \node[anchor=west] at ($(v_final.north west)+(0,0.4cm)$) {\textbf{Recombinaison V-(D)J}};

    %  Chaîne légère : segments germinaux 
    \foreach \name/\label/\i in {vl1/V1/0, vl2/V2/1, vl3/V3/2, vl4/V\dots/3, vln/Vn/4} {
        \node[rectangle, draw, fill=blue!5, rotate=90, minimum width=1.3cm, minimum height=0.8cm]
            (\name) at ($(0,-2.5)+(0,-1.4*\i)$) {\label};
    }

    \foreach \name/\label/\i in {jl1/J1/0, jl2/J2/1, jl3/J\dots/2, jln/JN/3} {
        \node[rectangle, draw, fill=orange!5, rotate=90, minimum width=0.6cm, minimum height=0.8cm]
            (\name) at ($(vln.east)+(0,-2)+(0,-0.8*\i)$) {\label};
    }

    \draw[thick] ($(vln.west)+(0,0)$) -- ($(jl1.east)+(0,0)$);

    \node[rectangle, draw, rotate=90, left=of jln, fill=gray!5, minimum width=3cm, minimum height=0.8cm] (cl) {C};
    \node[anchor=east, rotate=90] at ($(jl1.north west)+(-0.4,3cm)$) {\textbf{Segments germinaux - chaîne légère}};

    \node[rotate=90] at ($(vl1.east)+(0,0.3)$) {\textbf{5'}};
    \node[rotate=90] at ($(cl.west)+(0,-0.3)$) {\textbf{3'}};

    % Recombinaison V-J (chaine légère) 
    \node[rectangle, draw, right=3cm of cl, rotate=90, fill=gray!5, minimum width=3cm, minimum height=0.8cm] (cl_bis) {C};
    \node[rectangle, draw, right=of cl_bis, rotate=90, fill=orange!5, minimum width=0.6cm, minimum height=0.8cm] (jl_final) {J};
    \node[rectangle, draw, right=of jl_final, rotate=90, fill=red!5, dashed, minimum width=0.3cm, minimum height=0.8cm] (nl) {N};
    \node[rectangle, draw, right=of nl, rotate=90, fill=blue!5, minimum width=1.3cm, minimum height=0.8cm] (vl_final) {V};

    \draw[->, thick] (vl3.south) -- (vl_final.north);
    \draw[->, thick] (jl1.south) -- (jl_final.north);

    \node[anchor=west, rotate=90] at ($(cl_bis.north west)+(-0.4,0cm)$) {\textbf{Recombinaison V-J}};

    % Assemblage final : fragment d'immunoglobuline 
    \node[draw, fill=orange!30, rounded corners=8pt, rotate=45, minimum width=0.3cm, minimum height=1cm] at (7.5,-8) (j_ig) {J};
    \node[draw, fill=green!30, rounded corners=8pt, rotate=45, minimum width=0.3cm, minimum height=1cm, right=of j_ig] (d_ig) {D};
    \node[draw, fill=blue!20, rounded corners=8pt, rotate=45, minimum width=0.3cm, minimum height=1cm, right=of d_ig] (v_small_ig) {V};
    \node[draw, fill=blue!20, rounded corners=8pt, rotate=-45, minimum width=1.5cm, minimum height=1.3cm] at (8.8,-8.6) (v_ig) {VH};
    \node[draw, fill=gray!20, rounded corners=8pt, rotate=-45, minimum width=2cm, minimum height=1.3cm, right=of v_ig] (ch1) {CH1};
    \node[draw, fill=gray!20, rounded corners=8pt, rotate=-90, minimum width=2cm, minimum height=1.3cm] at (11,-12) (ch2) {CH2};
    \node[draw, fill=gray!20, rounded corners=8pt, rotate=-90, minimum width=2cm, minimum height=1.3cm, right=of ch2] (ch3) {CH3};
    \node[draw, fill=gray!5, below=0cm of ch1, rounded corners=8pt, rotate=-45, minimum width=2cm, minimum height=0.7cm] (clf) {CL};
    \node[draw, fill=blue!5, left=of clf, rounded corners=8pt, rotate=-45, minimum width=2.6cm, minimum height=0.7cm] (vj) {VL-JL};

    \draw[->, thick] (v_final.south) -- (v_small_ig.north);
    \draw[->, thick] (d_final.south) -- (d_ig.north);
    \draw[->, thick] (j_final.south) -- (j_ig.north);
    \draw[->, thick] (vl_final.south) -- (vj.west);
    \draw[->, thick] (jl_final.south) -- (vj.west);

    \end{tikzpicture}

    \caption{Étapes successives de la recombinaison \gls{vdj} des loci des immunoglobulines.
    V, D, J : segments \textbf{Variables}, \textbf{Diversité} et \textbf{Jonction} ; 
    C : régions \textbf{Constante} ; N : régions de \textbf{N-diversité} ;
    H : parties lourdes (\textbf{Heavy}) et L : parties légères (\textbf{Light}). 
    En haut : chaîne lourde ; à gauche : chaîne légère ; 
    en bas : schéma d'un fragment d'immunoglobuline complète.}
    \label{fig:vdj}
\end{figure}


Par conséquent, si l'on considère les loci des \glspl{ig}, la recombinaison \gls{vdj} permet de générer théoriquement : 

\begin{equation}
    \label{eq:combinaisons}
    \begin{aligned}
    \text{Chaîne lourde : } & \underbrace{50}_{\text{\# VH}} \times \underbrace{30}_{\text{\# DH}} \times \underbrace{6}_{\text{\# JH}} = 9000 \\
    \text{Chaîne légère } \kappa : & \underbrace{40}_{\text{\# Vk}} \times \underbrace{5}_{\text{\# Jk}} = 200 \\
    \text{Chaîne légère } \lambda : & \underbrace{46}_{\text{\# Vl}} \times \underbrace{7}_{\text{\# Jl}} = 322 \\
    \text{Total des combinaisons possibles : } & 9000 \times (200 + 322) = 4\,698\,000
    \end{aligned}
\end{equation}

Soit près de 5 millions de combinaisons possibles, en sachant que l'ajout des régions de diversité N
et des différentes régions constantes porte ce nombre à une valeur presque infinie (\autoref{eq:combinaisons}).

Ainsi, \textit{Vidjil} permet l'analyse des réarrangements des \glspl{ig} présents sur les loci suivants : 
\begin{itemize}
    \item \gls{igh} sur le chromosome 14, 
    \item \gls{igk} sur le chromosome 2, 
    \item \gls{igl} sur le chromosome 22.
\end{itemize}

Le répertoire des cellules T est également analysable, via les loci des \glspl{tcr} :
\begin{itemize}
    \item \gls{tra} et \gls{trd} sur le chromosome 14, 
    \item \gls{trb} et \gls{trg} sur le chromosome 7.
\end{itemize}

\vspace{1em}

Dans le cadre de ce travail, seules les recombinaisons des \gls{igh} seront considérées,
pour des raisons qui apparaîtront par la suite comme rapidement évidentes.
Si l'on s'intéresse plus en détail à la scructure des remaniements \gls{vdj} des \gls{igh},
on peut noter que les segments codant la région variable de la chaîne lourde (V, D et J)
incluent plusieurs sous-régions fonctionnelles. En amont du segment V, on trouve la région leader et 
le segment V lui-même est divisé en \glspl{fr} 1,2 et 3, qui assurent le repliement structural du domaine variable, 
(et servent de matrice pour la conception d'amorces d'amplification par PCR), 
et les \glspl{cdr}. Parmi celles-ci, la région \gls{cdr}3, située à la jonction des segments V, D et J 
est la plus variable et joue un rôle central dans la reconnaissance de l'antigène, et l'unicité du réarrangement.

\vspace{1em}

La diversité de cette région \gls{cdr}3 est générée par plusieurs mécanismes :
la recombinaison somatique entre les segments V, D et J, et l'ajout des domaines de N-diversité 
mentionnés précédemment, mais aussi l'excision de nucléotides en 3' et 5' des segments V, D et J.
À cela s'ajoute un processus appelé hypermutation somatique, qui affecte principalement le segment V 
après activation des lymphocytes B naïf, introduisant des mutations ponctuelles supplémentaires,
notamment dans les région \glspl{cdr}, afin d'augmenter l'affinité de l'anticorps pour son antigène (\autoref{fig:igh}).

\begin{figure}[H]
    \centering
    \begin{tikzpicture}[node distance=0cm, thick, font=\tiny]

    % Segments avec séquences nucléotidiques
    \node[rectangle, fill=blue!20, minimum width=6cm, minimum height=0.8cm, anchor=west] (v_full) at (0,0)
        {\shortstack{\textbf{VH3-30} \\ \dots ATGAATAGCCTGAGGCCT\textcolor{red}{G}AGGACACGGCTG\textcolor{red}{TG}TATTAC\textcolor{red}{T}GTAC\sout{A}}};

    \node[rectangle, right=of v_full, fill=red!20, dashed, minimum width=0.6cm, minimum height=0.8cm] (n1_full)
        {\shortstack{\textbf{N} \\ AAAGGGGTGGAC}};

    \node[rectangle, right=of n1_full, fill=green!20, minimum width=1.5cm, minimum height=0.8cm] (d_full)
        {\shortstack{\textbf{D3-3} \\ \sout{CG}ATGTTTGGAGT\sout{GGTT}}};

    \node[rectangle, right=of d_full, fill=red!20, dashed, minimum width=0.6cm, minimum height=0.8cm] (n2_full)
        {\shortstack{\textbf{N} \\ TCG}};

    \node[rectangle, right=of n2_full, fill=orange!20, minimum width=2cm, minimum height=0.8cm] (j_full)
        {\shortstack{\textbf{J4*02} \\ \sout{ACTGGGGCCAGG}GAACCCT\dots}};

    % Régions 3' et 5'
    \node[left=0.2cm of v_full.west] {\scriptsize \textbf{5'}};
    \node[right=0.2cm of j_full.east] {\scriptsize \textbf{3'}};

    % Régions fonctionnelles
    \draw[thick, blue] ($(v_full.west)+(-1,-0.6)$) -- ++(1,0) node[midway, below=2pt] {\scriptsize Leader};
    \draw[thick, blue] ($(v_full.west)+(0.7,-0.6)$) -- ++(1,0) node[midway, below=2pt] {\scriptsize FR1};
    \draw[thick, blue] ($(v_full.west)+(2,-0.6)$) -- ++(1,0) node[midway, below=2pt] {\scriptsize FR2};
    \draw[thick, blue] ($(v_full.west)+(4.7,-0.6)$) -- ++(1,0) node[midway, below=2pt] {\scriptsize FR3};

    \draw[thick, red] ($(v_full.east)+(-0.5,+0.6)$) -- ($(j_full.west)+(0.5,+0.6)$)
        node[midway, above=1.5pt] {\scriptsize \textbf{CDR3}};

    % Nom du réarrangement
    \node[below=1.5cm of n1_full.center, anchor=center, font=\large] {
        \textcolor{blue!60!black}{\texttt{IGHV3-30 1}}\,/%
        \textcolor{red!70!black}{\texttt{AAAGGGGTGGAC(12)}}\,/%
        \textcolor{green!70!black}{\texttt{2}}\,%
        \textcolor{green!70!black}{\texttt{D3-3 4}}\,/%
        \textcolor{red!70!black}{\texttt{TCG}}\,/%
        \textcolor{orange!80!black}{\texttt{12}}
        \textcolor{orange!80!black}{\texttt{J4*02}}
    };
    
    \end{tikzpicture}
    \caption{Illustration d'un réarrangement de la chaîne lourde des immunoglobulines \textit{\gls{igh}}.
    Les bases rayées correspondent aux nucélotides excisés lors de la recombinaison. 
    Les \textcolor{red}{bases en rouge} sont celles mutées par rapport aux séquences germinales lors du 
    processus d'hypermutation somatique.}
    \label{fig:igh}
\end{figure}


Chaque réarrangement \gls{vdj} peut être décrit à l'aide d'une nomenclature standardisée, 
permettant une identification unique et reproductible \cite{laneIMGTONTOLOGYIMGTLIGMotif2010,lefrancIMGT30Years2019}. 
Cette notation précise la nature du locus considéré, le segment V utilisé, les nucléotides excisés à son extrémité 3', 
la séquence ou longueur de la première région N insérée, le segment D (le cas présent) avec ses excisions en 3' et 5' 
et la seconde région N, ainsi que le segment J retenu et son excision en 5' (\autoref{tab:nomenclature-vdj}).

\begin{table}[H]
    \centering
    \caption{Nomenclature d’un réarrangement \gls{vdj}, BE : Bases excisées.}.
    \label{tab:nomenclature-vdj}
    \begin{tabular}{c c c c c c c c c c}
        \toprule
        \textbf{Locus} & \textbf{VH} & \textbf{BE V3'} & \textbf{N1} & \textbf{BE D5'} & 
        \textbf{D Nom} & \textbf{BE D3'} & \textbf{N2} & \textbf{BE J5'} & \textbf{J Nom} \\        
        \midrule IGH & VH3-30 & 1 & 12 & 2 & D3-3 & 4 & TCG & 12 & J4*02 \\
        \bottomrule
    \end{tabular}
\end{table}


\section{Applications en oncohématologie}

L'analyse du répertoire des gènes \gls{vdj} est particulièrement utile en oncohématologie, 
où la quasi unicité de chaque réarrangement \gls{vdj} peut servir de marqueur clonal hautement 
spécifique des cellules tumorales et ainsi permettre leur détection et leur suivi. 

Avant tout chose, il peut être utile de dresser un panorama rapide des hémopathies malignes 
pour saisir l'importance de l'analyse du répertoire des gènes \gls{vdj} dans certaines pathologies.
Ces pathologies regroupent ce qu'on appèle communément les « cancers du sang » et peuvent se séparer selon
leur origine en quatres grandes catégories. Une première séparation peut être établié entre les hémopathies

\section{Maladie résiduelle et ADNc}

\section{Problématique et objectifs}
