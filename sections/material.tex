\chapter{Matériels et Méthodes}

\section{Echantillons biologiques}

Les prélèvements sont issus d'une collaboration directe avec le laboratoire d'hématologie et la tumorothèque du 
\gls{chu} (GCS-ALLIANCE CANCER). Des échantillons de moelle osseuse et de sang périphérique de patients 
atteints de \gls{mm} et \gls{llc} ont été requalifiés à visée de recherche par le comité scientifique (n° d'avis CSTMT350 \autoref{anx:cstmt-mrd-mm}). 
Ces prélèvements ont été utilisés lors des étapes de mise au point technique. Par la suite 5 couples de prélèvements 
au diagnostic et au suivi de patients traités pour un \gls{mm} ont été sélectionnés pour l'analyse de la \gls{mrd}. Pour ces 
prélèvements, l'histoire clinique ainsi que la \gls{mrd} évalée par \gls{cmf} sont disponibles.

\section{Préparation et séquençage haut-débit}

Les échantillons de moelle osseuse et de sang périphérique sont traités selon les procédures standards du laboratoire 
d'hématologie : les prélèvements au diagnostic subissent un tri magnétique CD138+ pour isoler les plasmocytes puis l'ADN 
est extrait et cryopréservé. Pour les prélèvements au suivi, l'ADN est extrait directement à partir du culot cellulaire 
après lyse des hématies.

\vspace{1em}

La préparation de la librairie de séquençage est réalisée en une seule étape de \gls{pcr} via des amorces contenant les 
régions spécifiques à amplifier, les adaptateurs pour le séquençage et les codes-barres pour l'identification des échantillons 
(\autoref{anx:primer-sequences}).

\vspace{1em}

En raison de l'objectif d'analyser \textit{in fine} des échantillons d'\gls{adnc} qui, comme vu précédemment sont de petite taille, 
les amorces choisis ciblent la région \gls{fr}3 (cf \autoref{fig:igh}) des gènes \textit{\gls{igh}} pour pouvoir séquencer l'ensemble de la 
région \gls{cdr}3 porteuse de l'unicité du réarrangement. 
Ces amorces sont adaptées des recommandations du groupe européen de l'EuroClonalité \cite{langerakEuroClonalityBIOMED2Guidelines2012} 
et modifiées pour l'amplification d'\gls{adnc} par \citeauthor{pottCfDNABasedNGSIG2022a} \cite{pottCfDNABasedNGSIG2022a}.
Un standard interne (\textit{spike-in}) titré à $10^{-4}$ cellules est ajouté pour corriger les biais de séquençage en vue de la quantification 
pour les prélèvements au suivi. En parallèle, l'ADN du diagnostic est également utilisé dilué à $10^{-4}$ pour l'étude de la limite de détection. 
L'ADN amplifié est ensuite purifié sur billes, analysé et dosé puis séquencé sur plateforme Illumina MiSeq en \textit{paired-end} 2x300 bases.

\section{Environnement d'analyse et de développement}

L'analyse des données de séquençage est réalisée dans un premier temps sur \textit{Vidjil}, via une machine virtuelle dédiée,
hébergée au \gls{chu}. Par la suite, une fois les données anonymisées, elles sont transférées sur un environnement de 
développement Ubuntu 24.04.2 LTS (autrement dit, un ordinateur personnel) pour la mise au point de l'analyse de la \gls{mrd}.
\textit{Vidjil} est utilisé dans sa version 2024.02 pour \textit{vidjil-algo} et \texttt{release} 2024.12 puis 2025.06 pour la partie 
web. La configuration d'analyse utilisée est \textit{multi+inc+xxx} permettant la detection multi loci des recombinaisons complètes, 
incomplètes, atypiques et inattendues.
L'ensemble des modifications et développements réalisés sur \textit{Vidjil} concerne la partie client, et dans une moindre mesure,
la partie serveur. Elles sont consultables sur le dépôt \href{https://gitlab.inria.fr/users/x-benha/activity}{\textit{GitLab}}
de \textit{Vidjil}, et sont documentées dans les requêtes de fusion \href{https://gitlab.inria.fr/vidjil/vidjil/-/merge_requests/1610}{1610}
et \href{https://gitlab.inria.fr/vidjil/vidjil/-/merge_requests/1631}{1631} associées.

