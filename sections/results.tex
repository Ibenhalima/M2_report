\chapter{Résultats}

Préalablement aux résultats acquis durant le stage, plusieurs étapes de mises au point (notamment sur lignées cellulaires de plasmocytes et 
échantillons de \gls{llc}) ont été réalisées durant l'année 2024-2025. Les données ne seront pas présentés dans ce rapport, mais ont permis 
de valider la faisabilité de l'analyse des réarrangements \gls{vdj} via le protocole précédement décrit, sous-tenant quelques modifications, 
mais aussi de faire la lumière sur plusieurs difficultés.

\section{Limites et contraintes du protocole FR3}

\subsection{Identification incertaine du segment V}

Une des premières difficultés rencontrées, inhérente à la nature du protocole utilisé, est l'incertitude sur l'identification 
du segment V. En raison de l'utilisation d'amorces ciblant la région \gls{fr}3, toujours dans l'optique d'analyser par les suite 
des \gls{adnc}, seuls 20 à 30 nucléotides sont séquencés en amont du réarrangement \gls{vdj}. Cette longueur ne permet pas à 
\textit{Vidjil} (ou tout autre outil d'analyse de réarrangements \gls{vdj}) de déterminer avec certitude le segment V utilisé 
(\autoref{fig:v-leader-fr3}). 
Une façon de palier à ce problème est de séquençer au diagnostic en utilisant des amorces ciblant la région \gls{fr}1 ou 
\textit{leader}, pour obtenir une séquence plus longue, et ainsi permettre à \textit{Vidjil} de déterminer le segment V utilisé. 
L'unicité de la région \gls{cdr}3 permettra ensuite de relier les clones identifiés sur les prélèvements de suivi.

\begin{figure}[H]
    \centering
    \begin{ttfamily}
        \begin{tabular}{@{}l@{}}
        \textbf{Clone IGHV1-3*04 // D3-16 // J3*02 (\textit{leader})} \\
        \colorbox{blue!20}{GCCAGGCCCCCGGTCAGAGGCTTGAGTGGATGGGCTGGGTCAACGGTGCCAGTGGCGACGCAAAATATTCACAGCAT} \\
        \colorbox{blue!20}{TTCCAGGGCGGAGTCACCATTACCAGGGACACTTCCGCGACTACAGCCTACATGGAACTGAGCAGCCTGAGATCTGAG} \\
        \colorbox{blue!20}{GACACGGCTGTCTATTACTGTGCGA}%
        \colorbox{green!20}{CTTATACC}AACACTTTTTGGTT%
        \colorbox{orange!20}{TGCTTTTGATATCTGGGGCCAAGGGACAA} \\
        \colorbox{orange!20}{TGGTCACCGTCTCCTCAG}GT \\
        \\
        \textbf{Clone IGHV3-30*08 // D3-16 // J3*02 (\gls{fr}3)} \\
        \textbf{
            \textcolor{red}{\faExclamationTriangle\  Gènes V équiprobables : IGHV3-66*02, IGHV3-7*02, IGHV3-30*08, IGHV4-34*12}
            } \\
        \colorbox{blue!20}{GACACGGCTGTCTATTACTGTGCGA}%
        \colorbox{green!20}{CTTATACC}AACACTTTTTGGTT%
        \colorbox{orange!20}{TGCTTTTGATATCTGGGGCCAAGGGACAA} \\
        \colorbox{orange!20}{TGGTCACCGTCTCCTCAG}GT
        \end{tabular}
    \end{ttfamily}
    \caption{Alignement de deux réarrangements clonaux identiques. Le premier clone (en haut) correspond à un réarrangement 
    complet séquencé en \textit{leader}, tandis que le second (en bas) commence au niveau de la région \gls{fr}3. 
    Les séquences en \colorbox{blue!20}{bleu} correspondent au segment V, en \colorbox{green!20}{vert} 
    au segment D, et les séquences en \colorbox{orange!20}{orange} correspondent au segment J.
    Les jonctions sont alignées pour souligner la différence de longueur des régions V.}
    \label{fig:v-leader-fr3}
\end{figure}
 
\subsection{Impact de l'hypermutation somatique sur l'amplification}

Il s'agit encore une fois d'une difficulté liée au protocole d'amplification \gls{fr}3 et à la nature des cellules d'interêt. 
En effet les plasmocytes constituent le stade le plus mature de la lignée lymphoïde B, et en ce sens sont donc les cellules où le 
réarrangement \gls{vdj} comporte le moins d'homologie avec les séquence germinales. La conséquence de ceci étant que dans certains 
cas, les amorces utilisées pour l'amplification \gls{fr}3 ne sont pas capables de se fixer sur les séquences \gls{vdj} mutées, et 
amplifier les réarrangements. L'utilisation d'amorces dégénérées permet de limiter ce problème sans le résoudre totalement pour autant. 
Dans le cas ou le réarragenemt \gls{vdj} n'est pas amplifiable, il est possible d'analyser en lieu les réarrangements incomplets DH-JH 
(cf \autoref{fig:vdj}). Ainsi en guise d'illustration, parmi les 5 patients analysés, le patient 1 présente un réarrangement amplifié 
en \gls{fr}3, tandis que chez le patient 2, le réarrangement n'est pas amplifié (\autoref{fig:primer-alignement}).

\begin{figure}[H]
    \centering
    \begin{ColoredVerbatim}
                10         20 
        \G\Hbase\G\G\A\C\A\C\N\G\C\Y\G\T\G\T\A\T\T\A\C amorce dégénérée VH commune
        \textcolor{gray}{|:||||||:||:|||||||||}
        \G\A\G\G\A\C\A\C\G\G\C\T\G\T\G\T\A\T\T\A\C séquence IGHV3-30 patient 1
           150       160

              10        20 
        \G\G\A\C\A\C\N\G\C\Y\G\T\G\T\A\T\T\A\C amorce dégénérée VH commune
        \textcolor{gray}{|||| |:| :   ||||||}
        \G\G\A\C\Tb\C\A\G\G\C\A\C\Tb\T\A\T\T\A\C séquence IGHV2-5*04 patient 2
             160       170
    \end{ColoredVerbatim}
    \caption{
        Alignement de l'amorce dégénéree VH commune utilisé pour l'amplification \gls{fr}3 
        avec les séquences V majoritaires indentifiées chez les patients 1 et 2 en \textit{leader}. 
        Un alignement est représenté par | et : pour les nucléotides dégénerés. Les bases soulignées 
        correspondent aux bases mutées par rapport à la séquence germinale.
    }
    \label{fig:primer-alignement}
\end{figure}
    
\subsection{Amplification non spécifique et artefacts de PCR}

Un dernier problème, et non des moindre concerne l'amplification non spécifiques de régions non ciblées par les amorces, 
ainsi que la présence de nombreux artéfacts dans les données de séquençage. Les amorces utilisées faisant de l'ordre de 100 bases 
(cf \autoref{anx:primer-sequences}), elles ont tendance à se dimériser et ainsi être amplifiées et séquencées en l'état. 
De plus, certaines amorces, notamment celle ciblant la région JH commune, amplfient des régions non ciblées dans 
les conditions actuelles de \gls{pcr}. Pour illuster, chez le patient 1 sur le même prélèvement au diagnostic, on identifie 
un clone majoritaire qui represente près de 82 \% des clones identifiés en \textit{leader}, contre 32 \% en \gls{fr}3 au milieu 
de nombreuses séquences artéfactuelles et hors-cibles (\autoref{fig:fr3-vs-leader}).

\begin{figure}[H]
    \centering
    \includegraphics[width=1\textwidth]{images/diag_leader.png}
    \vspace{0.5cm}
    \includegraphics[width=1\textwidth]{images/diag_fr3.png}
    \caption{
        Répartition des clones identifiés en \textit{leader} (en haut) et en \gls{fr}3 (en bas) 
        pour le patient 1 au diagnostic. On observe que le \textcolor{Magenta}{clone majoritaire (en rose)} 
        en \textit{leader} devient plus minoritaire en \gls{fr}3, devant les \textcolor{ProcessBlue}{nombreux clones artéfactuels (en bleu)}  
        et \textcolor{ForestGreen}{hors-cibles (en vert)}.
    }
    \label{fig:fr3-vs-leader}
\end{figure}

\section{Optimisations techniques et bioinformatiques}

Un certain nombre des défis posés par le protocole de séquençage \gls{fr}3 ont ont pu être surmontés conjointement par 
des adapations techniques et bioinformatiques. Au niveau du protocole d'amplfication, une refonte complète des concentrations 
relative des différentes amorces, du protocole d'extration et des températures de \gls{pcr} ont permis d'améliorer nettement 
la qualité des données obtenues. Le cœur de cette section sera consacré au détail des approches bioinformatiques mises en place, 
notamment concernant les clones artéfactuels issus des dimères d'amorces et séquences hors-cibles.

\subsection{Filtrage des séquences hors-cibles}

En ce qui concerne les artéfacts issues des dimières d'amorces, il est assez naturel que \textit{Vidjil} les identifie comme 
des réarrangements \gls{vdj} valides, comprenat une section V et J. Il est par contre plus surprenant que des séquences ne comprenant 
ni gènes V ni J soient également identifiées comme des réarrangements \gls{vdj} valides. Ces séquences étant par ailleurs souvent majoritaires 
dans les premières données obtenues, pouvant représenter jusqu'à 50 \% \textit{reads} indentifiés par \textit{Vidjil}.

\vspace{1em}

Lorsqu'on aligne ces séquences sur le génome humain, on observe qu'elles correspondent à diverses régions du génome, avec un aligment quasi 
parfait, à l'exception de quelques bases aux extrémités des séquences. Ces bases sont toutes identiques et correspondent à la séquence 
\C\G\T\C\T\C\C\T\C\A\G\G\T\A\A\G, ou à sa séquence complémentaire (inversée) \C\T\T\A\C\C\T\G\A\G\G\A\G\A\C\G.

Dans l'hypothèse d'une amplification non spécifique, il est possible que ces séquences correspondent à des fragments d'amorces,
hypothèse qu'il est facile de vérifier en recherchant ces séquences dans les amorces utilisées pour l'amplification (\texttt{primers.fa} 
compremant les différentes combinaisons de bases dégénerées)
(\autoref{lst:bash-search-primer}).

\begin{lstlisting}[language=custombash, 
caption={Commande Bash et résultat de la recherche des séquences dans les amorces dégénérées.},
label={lst:bash-search-primer},
basicstyle=\ttfamily\scriptsize]
$ grep -i --color -C 1 "CTTACCTGAGGAGACG" primers.fa
>IGH-J-A-1
caagcagaagacggcatacagatxxxxxxxxgtgactggagttcagacgtgtgctcttccgatctCTTACCTGAGGAGACGgtgacc

$ echo "CGTCTCCTCAGGTAAG" | tr 'ACGT' 'TGCA' | rev | \
        xargs -I{} grep -i --color -C 1 {} primers.fa
>IGH-J-A-1
caagcagaagacggcatacagatxxxxxxxxgtgactggagttcagacgtgtgctcttccgatctCTTACCTGAGGAGACGgtgacc
\end{lstlisting}

\vspace{1em}

Ces séquences sont donc issues de l'amorce JH commune, et on peut en déduire le phénomène suivant :
dans les conditions de \gls{pcr} utilisées (nottement de température), l'amorce JH se fixe à des régions 
du génome autres que les régions JH par complémentarité faible, et amplfie ces régions recopiant au passage 
la séquence de l'amorce spécfique de la région JH (\autoref{fig:off-target-amplification}). Ces séquences sont ensuite 
identifiées par \textit{Vidjil} comme des réarrangements \gls{vdj} valides incomplets. 

\begin{figure}[H]
    \centering
    \begin{ColoredVerbatim}
        
        NNNNNNNNNNNNNNNNNNNNNNNNNNNNNNNNNNNNNNNN Séquence quelconque double brin
        NNNNNNNNNNNNNNNNNNNNNNNNNNNNNNNNNNNNNNNN

        NNNNNNNNNNNNNNNNNNNNNNNNNNNNNNNNNNNNNNNN Dénaturation (Cycle 1)

                               \C\G\T\C\T\C\C\T\C\A\G\G\T\A\A\G
        NNNNNNNNNNNNNNNNNNNNNNNNNNNNNNNNNNNNNNNN Fixation de l'amorce JH


        NNNNNNNNNNNNNNNNNNNNNNN\C\G\T\C\T\C\C\T\C\A\G\G\T\A\A\G
        NNNNNNNNNNNNNNNNNNNNNNNNNNNNNNNNNNNNNNNN Amplification

        NNNNNNNNNNNNNNNNNNNNNNN\C\G\T\C\T\C\C\T\C\A\G\G\T\A\A\G Dénaturation (Cycle 2)

                               \C\G\T\C\T\C\C\T\C\A\G\G\T\A\A\G
        NNNNNNNNNNNNNNNNNNNNNNNNNNNNNNNNNNNNNNNN Fixation de l'amorce JH, etc.
    \end{ColoredVerbatim}
    \caption{
        Représentation schématique de l'amplification non spécifique des séquences hors-cibles. 
        La séquence \C\G\T\C\T\C\C\T\C\A\G\G\T\A\A\G est issue de l'amorce JH commune, et est recopiée 
        lors de l'amplification des séquences hors-cibles.
    }
    \label{fig:off-target-amplification}
\end{figure}

Il est intéressant de noter que seule la séquence sens \C\G\T\C\T\C\C\T\C\A\G\G\T\A\A\G\ est retrouvée dans les fichiers de \textit{reads} 
R1, tandis que la séquence anti-sens \C\T\T\A\C\C\T\G\A\G\G\A\G\A\C\G\ uniquement dans les fichiers de \textit{reads} 
R2 (\autoref{lst:bash-primer-r1-r2}). On peut donc en déduire que seule la séquence sens de l'amorce JH se fixe sur 
le brin d'ADN et n'est jamais recopiée, confortant l'hypothèse formulée.

\begin{lstlisting}[language=custombash, 
    caption={Commande Bash et résultat de la recherche des séquences des amorces dans les fichiers FASTQ R1 et R2.},
    label={lst:bash-primer-r1-r2},
basicstyle=\ttfamily\small]
# amorce sens
$ zgrep -c "CGTCTCCTCAGGTAAG" FR3_P01-TEMOIN_PBL_IVS_S7_R1_001.fastq.gz
147224
$ zgrep -c "CGTCTCCTCAGGTAAG" FR3_P01-TEMOIN_PBL_IVS_S7_R2_001.fastq.gz
0

# amorce anti-sens
$ zgrep -c "CTTACCTGAGGAGACG" FR3_P01-TEMOIN_PBL_IVS_S7_R2_001.fastq.gz
159060
$ zgrep -c "CTTACCTGAGGAGACG" FR3_P01-TEMOIN_PBL_IVS_S7_R1_001.fastq.gz
0
\end{lstlisting}

\vspace{1em}
    
Par ailleurs ces clonotypes ne resprésentent que la fasse visible du problème, car en réalité, un certain nombre de \textit{reads} 
hors-cibles ne sont pas identifiés comme des réarrangements \gls{vdj} valides par \textit{Vidjil}, mais sont pour autant présents 
dans les données de séquençage. Il est possible de  se mesurer l'ampleur du phénomène en alignant les fichiers FASTQ de séquençage.

\vspace{1em}

cf données figure CHR alignement.

\vspace{1em}

Pour pallier à ce problème, il est possible de filtrer les séquences en ajustant les paramètres de \textit{vidjil-algo}.
Pour chaque clone, une \textit{e-value} (valeur E) est calculée, souvent de l'orde de $10^{-50}$ à $10^{-100}$, dans le 
cadre de vrais réarrangements \gls{vdj}. En revanche, pour les séquences hors-cibles, cette valeur est de l'ordre de $10^{-5}$ à $10^{-10}$. 
Ainsi, en filtrant les séquences avec une \textit{e-value} inférieure à $10^{-15}$, on peut éliminer la plupart des séquences hors-cibles, et 
il suffit pour cela de modifier le paramètre \texttt{-e} contrôlant la \textit{e-value} maximale autorisée de la configuration \textit{multi+inc+xxx} 
de \textit{vidjil-algo} correspondant à l'instruction \texttt{-c clones -z 100 -r 1 -g germline/homo-sapiens.g \textcolor{red}{-e 1} -2 -d -w 50} 
en \textcolor{red}{-e 1e-15} (\autoref{fig:fr3-evalue}).

\begin{figure}[H]
    \centering
    \includegraphics[width=1\textwidth]{images/diah_fr3_e1.png}
    \vspace{0.5cm}
    \includegraphics[width=1\textwidth]{images/diag_fr3_e-15.png}
    \caption{
        Répartition des clones identifiés en \gls{fr}3 avec une \textit{e-value} maximale de 1 (en haut) 
        et de $10^{-15}$ (en bas) pour le patient 1 au diagnostic. \textcolor{Magenta}{Clone majoritaire (en rose)}, 
        \textcolor{ProcessBlue}{clones artéfactuels (en bleu)} et \textcolor{ForestGreen}{hors-cibles (en vert)}.
    }
    \label{fig:fr3-evalue}
\end{figure}

\subsection{Supression réversible de clonotypes}

Ainsi, une simple adaptation de la configuration de \textit{vidjil-algo} a permis de réduire significativement la détection des clonotypes 
artéfactuels liés aux séquences hors-cibles. Pour autant, certains de ces clonotypes persistent dans les résultats, tandis que les dimères 
d'amorces demeurent en grande partie non filtrés. Ces artéfacts fausseraient dès lors toute quantification de la \gls{mrd} en sous-estimant 
le poids réels des véritables clones.
Il est possible via l'interface web de \textit{Vidjil} de masquer temporairement certains 
clonotypes qui restent alors comptabilisés dans les clones analysés, mais redeviennent visible à chaque rafraichissement de la page sans 
possibilité de sauvegarde. Il a donc été nécessaire de développer une fonctionnalité permettant la suppression contrôlée des clonotypes, 
répondant également aux attentes exprimées par d'autres utilisateurs.

\vspace{1em}

\textbf{Exigences fonctionnelles définies pour le développement} :
\begin{itemize}
    \item Permettre à l'utilisateur de supprimer un ou plusieurs clonotypes de l'analyse.
    \item Rendre la suppression réversible, avec possibilité de restauration des clonotypes supprimés.
    \item Enregistrer les clonotypes supprimés lors de la sauvegarde des résultats.
    \item Exclure les clonotypes supprimés du total des clones analysés.
    \item Afficher de façon claire et explicite les clonotypes supprimés à l'utilisateur.
    \item Ne pas représenter les clonotypes supprimés dans les graphiques de répartition des clones.
\end{itemize}

\vspace{1em}

La mise en œuvre de ce cahier des charges implique principalement des modifications de la partie client de \textit{Vidjil}, dont il est 
pertinent d'en présenter brièvement l'architecture avant de détailler la solution implémentée. Cette partie client est écrite en JavaScript natif 
et repose sur 58 scripts, dont 2 sont particulièrement importants pour cette nouvelle fonctionnalité : \texttt{model.js} et \texttt{clone.js}. 
Le premier gère via un objet \texttt{model} l'ensemble des attributs et méthodes nécessaires à la gestion synchrone des données des clonotypes 
et métadonnées d'analyses, tandis que le second gère des objets \texttt{clone} comprenant les données et méthodes propres à chaque clonotype. 

\vspace{1em}

L'idée est donc la suivante : un nouvel attribut booléen \texttt{removed} (initialisé par défaut sur la valeur \texttt{false}) est ajouté à l'objet 
\texttt{clone} pour indiquer si le clonotype doit être supprimé ou non. Cet attribus est contrôlé dans l'implémentation actuelle par un nouveau 
tag utilisateur \texttt{removed\ clonotype}. Deux attributs \texttt{removed\_clones\_reads\_of\_active\_locus} et \texttt{removed\_clones\_reads\_total} sont 
ajoutés à l'objet \texttt{model} pour stocker le nombre de \textit{reads} des clonotypes supprimés indéxé par le temps sous forme d'\textit{array} (tableau), 
respectivement pour le locus actif et pour l'ensemble des loci (\autoref{algo:removed-reads}). 
Enfin la méthode \texttt{getSize} de l'objet \texttt{clone} est modifiée pour soustraire du total des \textit{reads} le nombre de \textit{reads} des clonotypes 
supprimés, ainsi que d'autres indicateurs de nombre de \textit{reads} analysés.

\vspace{1em}

\begin{algorithm}[H]
    \caption{Calcul des \textit{reads} des clonotypes supprimés à chaque temps}
    \label{algo:removed-reads}
    \KwIn{Nombre total de points temporels $T$}
    \KwOut{Vecteurs des \textit{reads} supprimés pour le locus sélectionné et au total}
    
    \SetKwData{RemovedReadsLocus}{removed\_clones\_reads\_of\_active\_locus}
    \SetKwData{RemovedReadsTotal}{removed\_clones\_reads\_total}
    \SetKwData{Germline}{germline}
    
    \tcp{Initialisation des variables}
    \RemovedReadsLocus $\leftarrow$ tableau de taille $T$ initialisé à $0$\;  
    \RemovedReadsTotal $\leftarrow$ tableau de taille $T$ initialisé à $0$\;  
    \Germline $\leftarrow$ liste vide\;  
    
    \tcp{Récupération des loci sélectionnés}
    \For{chaque $i$ dans \texttt{system\_selected}}{
        Ajouter $i$ à \Germline\; 
    }
    
    \tcp{Parcours des clones et mise à jour des reads supprimés}
    \For{chaque $j$ dans \texttt{clones}}{
        \If{\texttt{clones[$j$]} est marqué comme supprimé}{
            \For{$t$ de $0$ à $T-1$}{
                $reads \leftarrow$ \texttt{clones[$j$].getReads($t$)}\; \tcp{Reads du clone $j$ au temps $t$}
                \RemovedReadsTotal$[t] \mathrel{+}= reads$\;
                \If{\Germline contient \texttt{clones[$j$].get('germline')}}{
                    \RemovedReadsLocus$[t] \mathrel{+}= reads$\;
                }
            }
        }
    }
    
    Mise à jour des attributs \RemovedReadsLocus et \RemovedReadsTotal\
    \end{algorithm}

\vspace{1em}

En parallèle, des modifications de l'interface utilisateur ont été réalisées afin d'afficher les clonotypes supprimés avec un format distinct 
permettant de les différencier des autres. Ces clonotypes sont ainsi représentés en transparence et barrés (\autoref{fig:removed-clonotypes}).  
Un clonotype virtuel est également ajouté pour chaque locus afin de comptabiliser le pourcentage de \textit{reads} supprimés par locus.  
Les clonotypes peuvent être restaurés en supprimant le tag \texttt{removed clonotype}, cette information étant conservée via ce même tag.  
Enfin, des tests unitaires ont été ajoutés pour garantir le bon fonctionnement de cette fonctionnalité.

\begin{figure}[H]
    \centering
    \includegraphics[width=0.7\textwidth]{images/removed_clonotypes.png}
    \caption{
        Capture d'écran de l'interface utilisateur de \textit{Vidjil} montrant les clonotypes supprimés. 
        Les clonotypes sont affichés en transparence et barrés, et un clonotype virtuel est ajouté pour 
        comptabiliser le pourcentage de \textit{reads} supprimés par locus.
    }
    \label{fig:removed-clonotypes}
\end{figure}

\section{Etude de la MRD}

Il nous est maintenant possible de supprimer les clonotypes artéfactuels et hors-cibles pour ne conserver que les réarrangements véritables lors de 
l'étude de la \gls{mrd}. L'évaluation de la \gls{mrd} est réalisée en comparant les réarrangements clonaux identifiés au diagnostic à ceux présents 
au suivi, et nécessite de pouvoir détecter avec précision des quantité très faibles de réarrangements clonaux.

\subsection{Quantité de matériel nécessaire selon le seuil de détection analytique}

En premier lieu, il est nécessaire de s'assurer que le matériel de départ est suffisant pour espérer détecter des quantités infimes de réarrangements 
clonaux. Dans cette optique, il est possible de modéliser de façon assez simple la question par un problème d'échantillionnage statistique. Si l'on considère 
l'organisme comme la population ou chaque cellule représente un individu, positif ou négatif (soit une cellule tumorale ou saine), avec une probabilité $p$ 
qu'une cellule soit positive, en considérant chaque cellule comme un échantillon indépendant, le nombre $N$ de cellules positives parmis $n$ cellules analysés 
suivra une loi binomiale de paramètres $n$ et $p$. Dans les faits, sachant que $p$ est très faible, on peut approximer la loi binomiale par une loi de Poisson de 
paramètre $\lambda = n \cdot p$. 

\vspace{1em}

Ainsi, si l'on accepte un risque $\epsilon$ de ne pas détecter un réarrangement clonal, en analysant $n$ cellules, malgré que le réarrangement soit présent chez le 
patient, on peut écrire comme suit :

\begin{equation}
    N \sim \mathcal{B}(n, p) \approx \mathcal{P}(\lambda)
    \quad \text{avec} \quad \lambda = n \cdot p
\end{equation}

La probabilité de détecter au moins une cellule tumorale est donnée par :
\begin{equation}
    \mathbb{P}(N \geq 1) = 1 - \mathbb{P}(N = 0) = 1 - e^{-\lambda} = 1 - \varepsilon
\end{equation}

où $\varepsilon$ est le risque de non-détection fixé à l'avance (par exemple $5\,\%$ pour un seuil de confiance de $95\,\%$).
En pratique, la sensibilité analytique des techniques de mesure de la \gls{mrd} ne permettent pas la détection d'un événement unique, et il est plus juste de placer 
un seuil de détection minimal $k$ de l'ordre de 10 à 100 cellules positives : 

\begin{equation}
    \mathbb{P}(N \geq k) = 1 - \sum_{i = 0}^{k-1} \frac{e^{-\lambda} \, \lambda^{i}}{i!} = 1 - \varepsilon
\end{equation}

On cherche donc a résoudre l'équation précédente en $n$, en sachant que $p$ est inconnu en pratique, mais estimé comme étant de l'ordre de $10^{-4}$ à $10^{-8}$, 
valeur qui correspondrait à la proportion des cellules dites « initiatrices de leucémie ». En faisant donc varier $p$ sur cet intervalle, avec $\epsilon = 0.05$ et 
$k = 10$ puis $100$, on peut résoudre l'équation précendente (de façon numérique et non algébrique, s'agissant d'une équation transcendante). En considérant également 
la numération leucocytaire sanguine comme étant de l'ordre de $5\,\text{G/L}$ et médullaire de $50\,\text{G/L}$, on peut en déduire le nombre minimal de cellules sanguines à analyser 
pour espérer détecter la \gls{mrd} au seuil de confiance de $95\,\%$ (\autoref{tab:valeurs_n_volume}).

\begin{table}[H]
    \centering
    \caption{Nombre minimal de cellules $n$ et volume sanguin nécessaire pour différents $p$ et seuils $k$, avec $\varepsilon = 0.05$}
    \label{tab:valeurs_n_volume}
    \begin{tabular}{c|cc|cc}
        \toprule
        \multirow{2}{*}{$p$} & \multicolumn{2}{c|}{$k = 10$} & \multicolumn{2}{c}{$k = 100$} \\
        & $n$ (cellules) & Volume (mL) & $n$ (cellules) & Volume (mL) \\
        \midrule
        \multicolumn{5}{c}{Numération = $5\,\text{G/L}$} \\
        \midrule
        $1 \times 10^{-4}$ & $1.57 \times 10^{5}$ & 0.031 & $1.17 \times 10^{6}$ & 0.234 \\
        $1 \times 10^{-5}$ & $1.57 \times 10^{6}$ & 0.314 & $1.17 \times 10^{7}$ & 2.340 \\
        $1 \times 10^{-6}$ & $1.57 \times 10^{7}$ & 3.14 & $1.17 \times 10^{8}$ & 23.40 \\
        $1 \times 10^{-7}$ & $1.57 \times 10^{8}$ & 31.43 & $1.17 \times 10^{9}$ & 234.00 \\
        $1 \times 10^{-8}$ & $1.57 \times 10^{9}$ & 314.29 & $1.17 \times 10^{10}$ & 2340.00 \\
        \midrule
        \multicolumn{5}{c}{Numération = $50\,\text{G/L}$} \\
        \midrule
        $1 \times 10^{-4}$ & $1.57 \times 10^{5}$ & 0.0031 & $1.17 \times 10^{6}$ & 0.023 \\
        $1 \times 10^{-5}$ & $1.57 \times 10^{6}$ & 0.031 & $1.17 \times 10^{7}$ & 0.234 \\
        $1 \times 10^{-6}$ & $1.57 \times 10^{7}$ & 0.31 & $1.17 \times 10^{8}$ & 2.34 \\
        $1 \times 10^{-7}$ & $1.57 \times 10^{8}$ & 3.14 & $1.17 \times 10^{9}$ & 23.40 \\
        $1 \times 10^{-8}$ & $1.57 \times 10^{9}$ & 31.43 & $1.17 \times 10^{10}$ & 234.00 \\
        \bottomrule
    \end{tabular}
\end{table}

Ainsi, pour un seuil de détection à $10^{-5}$ avec 100 cellules minimum, il est nécessaire de prélever au moins 3 mL de sang 
et 10 fois moins de moelle, ce qui correspond à une quantité d'environ 500 ng d'ADN (A vérfier avec Aurélie). On remarque également 
que des seuils de détection inférieurs à $10^{-7}$ sont jusqu'à présent inatteignable, nécessaitant des volumes de prélèvement pouvant 
attiendre plusieurs litres.

TO BE CONTINUED

\subsection{Quantification de la MRD}

\subsection{Détection de la MRD sur les échantillons cliniques}