\chapter{Résultats}

Préalablement aux résultats acquis durant le stage, plusieurs étapes de mises au point (nottament sur lignées cellulaires de plasmocytes) 
ont été réalisées durant l'année 2024-2025. Les données ne seront pas présentés dans ce rapport, mais ont permis de valider la faisabilité 
de l'analyse des réarrangements \gls{vdj} via le protocole précédement décrit, sous-tenant quelques modifications, mais aussi de faire 
la lumière sur plusieurs difficultés.

\section{Difficultés techniques liées au protocole de séquençage FR3}

\subsection{Identification du segment V}

Une des premières difficultés rencontrées, inhérente à la nature du protocole utilisé, est l'incertitude sur l'identification 
du segment V. En raison de l'utilisation d'amorces ciblant la région \gls{fr}3, toujours dans l'optique d'analyser par les suite 
des \gls{adnc}, seuls 20 à 30 nucléotides sont séquencés en amont du réarrangement \gls{vdj}. Cette longueur ne permet pas à 
\textit{Vidjil} (ou tout autre outil d'analyse de réarrangements \gls{vdj}) de déterminer avec certitude le segment V utilisé 
(\autoref{fig:v-leader-fr3}). 
Une façon de palier à ce problème est de séquençer au diagnostic en utilisant des amorces ciblant la région \gls{fr}1 ou 
\textit{leader}, pour obtenir une séquence plus longue, et ainsi permettre à \textit{Vidjil} de déterminer le segment V utilisé. 
L'unicité de la région \gls{cdr}3 permettra ensuite de relier les clones identifiés sur les prélèvements de suivi.

\begin{figure}[H]
    \centering
    \begin{ttfamily}
        \begin{tabular}{@{}l@{}}
        \textbf{Clone IGHV1-3*04 // D3-16 // J3*02 (\textit{leader})} \\
        \colorbox{blue!20}{GCCAGGCCCCCGGTCAGAGGCTTGAGTGGATGGGCTGGGTCAACGGTGCCAGTGGCGACGCAAAATATTCACAGCAT} \\
        \colorbox{blue!20}{TTCCAGGGCGGAGTCACCATTACCAGGGACACTTCCGCGACTACAGCCTACATGGAACTGAGCAGCCTGAGATCTGAG} \\
        \colorbox{blue!20}{GACACGGCTGTCTATTACTGTGCGA}%
        \colorbox{green!20}{CTTATACC}AACACTTTTTGGTT%
        \colorbox{orange!20}{TGCTTTTGATATCTGGGGCCAAGGGACAA} \\
        \colorbox{orange!20}{TGGTCACCGTCTCCTCAG}GT \\
        \\
        \textbf{Clone IGHV3-30*08 // D3-16 // J3*02 (\gls{fr}3)} \\
        \textbf{
            \textcolor{red}{\faExclamationTriangle\  Gènes V équiprobables : IGHV3-66*02, IGHV3-7*02, IGHV3-30*08, IGHV4-34*12}
            } \\
        \colorbox{blue!20}{GACACGGCTGTCTATTACTGTGCGA}%
        \colorbox{green!20}{CTTATACC}AACACTTTTTGGTT%
        \colorbox{orange!20}{TGCTTTTGATATCTGGGGCCAAGGGACAA} \\
        \colorbox{orange!20}{TGGTCACCGTCTCCTCAG}GT
        \end{tabular}
    \end{ttfamily}
    \caption{Alignement de deux réarrangements clonaux identiques. Le premier clone (en haut) correspond à un réarrangement 
    complet séquencé en \textit{leader}, tandis que le second (en bas) commence au niveau de la région \gls{fr}3. 
    Les séquences en \colorbox{blue!20}{bleu} correspondent au segment V, en \colorbox{green!20}{vert} 
    au segment D, et les séquences en \colorbox{orange!20}{orange} correspondent au segment J.
    Les jonctions sont alignées pour souligner la différence de longueur des régions V.}
    \label{fig:v-leader-fr3}
\end{figure}
 
\subsection{Hypermutation somatique}

Il s'agit encore une fois d'une difficulté liée au protocole d'amplification \gls{fr}3 et à la nature des cellules d'interêt. 
En effet les plasmocytes constituent le stade le plus mature de la lignée lymphoïde B, et en ce sens sont donc les cellules où le 
réarrangement \gls{vdj} comporte le moins d'homologie avec les séquence germinales. La conséquence de ceci étant que dans certains 
cas, les amorces utilisées pour l'amplification \gls{fr}3 ne sont pas capables de se fixer sur les séquences \gls{vdj} mutées, et 
amplifier les réarrangements. L'utilisation d'amorces dégénérées permet de limiter ce problème sans le résoudre totalement pour autant. 
Dans le cas ou le réarragenemt \gls{vdj} n'est pas amplifiable, il est possible d'analyser en lieu les réarrangements incomplets DH-JH 
(cf \autoref{fig:vdj}). Ainsi en guise d'illustration, parmi les 5 patients analysés, le patient 1 présente un réarrangement amplifié 
en \gls{fr}3, tandis que chez le patient 2, le réarrangement n'est pas amplifié (\autoref{fig:primer-alignement}).

\begin{figure}[H]
    \centering
    \begin{ColoredVerbatim}
                10         20 
        \G\Hbase\G\G\A\C\A\C\N\G\C\Y\G\T\G\T\A\T\T\A\C amorce dégénérée VH commune
        \textcolor{gray}{:.::::::.::.::::::::}
        \G\A\G\G\A\C\A\C\G\G\C\T\G\T\G\T\A\T\T\A\C séquence IGHV3-30 patient 1
           150       160

             10        20 
        \G\G\A\C\A\C\N\G\C\Y\G\T\G\T\A\T\T\A\C amorce dégénérée VH commune
        \textcolor{gray}{:::: :.: .   ::::::}
        \G\G\A\C\Tb\C\A\G\G\C\A\C\Tb\T\A\T\T\A\C séquence IGHV2-5*04 patient 2
             160       170
    \end{ColoredVerbatim}
    \caption{
        Alignement de l'amorce dégénéree VH commune utilisé pour l'amplification \gls{fr}3 
        avec les séquences V majoritaires indentifiées chez les patients 1 et 2 en \textit{leader}. 
        Un alignement est représenté par : et . pour les nucléotides dégénerés. Les bases soulignées 
        correspondent aux bases mutées par rapport à la séquence germinale.
    }
    \label{fig:primer-alignement}
    \end{figure}
    
\subsection{Amplification non spécifique}

Un dernier problème, et non des moindre concerne l'amplification non spécifiques de régions non ciblées par les amorces, 
ainsi que la présence de nombreux artéfacts dans les données de séquençage. Les amorces utilisées faisant de l'ordre de 100 bases 
(cf \autoref{anx:primer-sequences}), elles ont tendance à se dimériser et ainsi être amplifiées et séquencées en l'état. 
De plus, certaines amorces, nottament celle ciblant la région JH commune, on tendance à se fixer sur des régions non ciblées et
à être amplifiées. Elles sont 



\section{Développements }

Un certain nombre des défis posés par le protocole de séquençage \gls{fr}3 ont ont pu être surmontés conjointement par 
des adapations techniques et bio-informatiques. Au niveau du protocole d'amplfication, une refonte complète des concentrations 
relative des différentes amorces, du protocole d'extration et des températures de \gls{pcr} ont permis d'améliorer nettement 
la qualité des données obtenues. 